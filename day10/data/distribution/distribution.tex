\begin{problem}{Distribution}
{stdin}{stdout}{2 second}{1024 megabytes}

令$p$是一个$(0,1)$之间的实数,$n$是一个$[1,100]$之间的正整数。

令$X_1, X_2, \dots, X_n$为$n$个独立的随机变量,并且满足$\Pr(X_i=1)=p, \Pr(X_i=0)=1-p$,考虑下面的随机变量

$$X=\frac{X_1+X_2+\dots + X_n +u -np}{\sqrt{np(1-p)}}$$

其中$u$是在$[-0.5,0.5]$之间均匀分布的实数随机变量,与$Xi$独立。

现在对于某个$n$给出了很多个$X$的取值,想让你猜出$n$的值。

\Input

第一行一个整数$T$,表示数据组数。接下来一行一个实数$p$在$(0,1)$之间,小数点之后至多两位。

接下来$T$行,每行开始一个整数$N$,接下来$N$个实数,表示随机出来的若干个$X$的取值。

保证这里隐藏的$n$也是在$[1,100]$之间随机生成的。

\Output

共$T$行,每行一个整数,表示你猜出来的$n$的取值。

\Example
\begin{example}
\exmpfile{/distribution/ex_distribution1.in}{/distribution/ex_distribution1.ans} %
\end{example}

\Constraints

对于$100\%$的数据,$T=30, N=10^4$,其中第$i$组数据满足$p=0.05+0.09*i$。

\Scoring

令$s=\sum |your_i-answer_i|$,也就是你的答案和标准答案的差的总和。

如果$s\leq 5$,你将获得$10$分。

如果$s\leq 15$,你将获得$9$分。

如果$s\leq 150$,你将获得$7$分。

如果$s\leq 300$,你将获得$5$分。

如果$s\leq 600$,你将获得$3$分。

\end{problem}